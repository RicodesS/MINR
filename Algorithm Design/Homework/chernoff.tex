 \documentclass[10pt]{article}

\usepackage{amsmath,amsthm,amsfonts}
\usepackage{graphicx}
\usepackage{enumerate}
\usepackage{xcolor}
\usepackage{ulem}

\newtheorem{theorem}{Theorem}

\begin{document}
Chernoff bounds are based on the markov's inequality (it gives a bound on the expectation): for any $a > 0,$
$$
\Pr(X > a) \leq \frac{\E(X)}{a}.
$$\\ the bounds' theorem follows: $
    Let $X = \sum_{i=1}^n X_i$, where $X_i = 1$ with probability $p_i$ and $X_i = 0$ with probability $1-p_i$, and all $X_i$ are independent.
    Let $\mu = \E(X) = \sum_{i=1}^n p_i.$ Then
	\Pr(X > (1 + \delta)\mu) \leq  e^{-\frac{\delta^2}{2 + \delta} \mu}$ for all $\delta > 0$;
